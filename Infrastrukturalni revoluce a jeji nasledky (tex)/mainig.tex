\documentclass{article}
\usepackage{fullpage}
\usepackage[czech]{babel}
\usepackage{amsfonts}
\renewcommand{\baselinestretch}{1.5}
\title{\vspace{-2cm}Proč je důležitá dopravní infrastruktura pro růst ekonomiky a co znamenaly změny zátěžových směrů v Československu\vspace{-1.7cm}}
\date{}
\author{}

\begin{document}
\maketitle
\tableofcontents
\newpage
\section{Úvod}

\paragraph{\quad Doprava je každodenní součástí našich životů již nelze přehlédnout. Co se, ale skrývá za našimi zraky a jak je možné, že všechny produkty či lidé jsou tam kde chtějí být ve správný čas? Mohl by svět, jak ho známe fungovat bez tak rozvinuté infrastruktury? Odpověď je jednoduchá, nemohl. Infrastruktura se vyvíjí prakticky od počátku lidstva. Každá vyspělá společnost si zakládá na efektivní, vyspělé ekonomice. Za efektivní ekonomiku považujeme ekonomiku, která je schopna využít všech svých statků jež má k dispozici bez toho, aniž by obohacovala jednoho jedince na úkor někoho jiného. Jejíž základem jsou tři otázky co, jak a pro koho. Avšak bylo by záhodno zde přidat i otázky kam nebo čím. Upozaděn je totiž fakt, že do továren a obchodů musí být suroviny, produkty nebo pracovní síla nějak doručena. Mezi nabídkou a poptávkou nacházíme tedy třetí neutrální stranu: logistiku. To už však věděli i staří Římané. Římské impérium by se svým územím přes celou Evropu bez, na jeho dobu velice vyspělé, sítě silnic nemohlo fungovat. Pokud nahlédneme dále do historie můžeme si všimnout jistého válečného trendu. Vyhrává ten, kdo má fungující logistiku nebo ochromí logistiku nepřítele. Toto jsme mohli pozorovat nejen u Římanů, ale i Francie, Pruska, Anglie potom za druhé světové války při velkých bitvách na moři o zásobování, jenž nakonec způsobily pád Německa. Zkrátka není vyspělé ekonomiky bez vyspělé infrastruktury. Bohužel to stále spoustě lidem nedochází.
 }
\paragraph{\quad Tato práce se }


\section{1. období (1918-1937)}
\subsection{Národně-hospodářský kontext}

\paragraph{\quad Rozpad Rakouska-Uherska znamenal pro střední Evropu spoustu nových problémů, ale i příležitostí. Nejen, že došlo k novému mezinárodně-politickému uspořádání střední Evropy, ale zároveň to též způsobilo rozpad hospodářských vztáhů jak ve střední Evropě, tak v Evropě samotné. To dalo za vznik těžkému ekonomickému nacionalismu a uzavřenosti zemí střední Evropy. Českloslovensko mělo tu výhodu, že patřilo k dohodovým státům, a tak i přes všechny dohady ohledně území, rychle nabylo politické stability. Rozpad Rakouska-Uherska spíše poznamenal země průmyslové, tedy Československo a Rakousko. Po rozpadu Československo získalo 21\% území tehdejšího Rakouska-Uherska, ale na jeho území se nacházelo až 70\% průmyslové kapacity bývalé monarchie. Avšak kontraintuitivně většina podnikatelů a národohospodářů prosazovala co nejrychlejší hospodářské osamostatnění vůči bývalým státům monarchie.
\\\indent Jedním z hlavních propagátorů tohoto názoru byl nový ministr financí Alois Rašín, jenž vstoupil do funkce v roce 1918. V této době jako ministr financí též zaujímal post guvernéra národní banky, což mu otevřelo mnoho možností k regulování státního rozpočtu, vnitrostátního obchodu a zahraničního obchodu. Významně přispěl při vytváření samostatné československé ekonomiky zavedením Měnové odluky a československé koruny. Měnová odluka spočívala ve okolkování starých rakousko-uherských bankovek. Původně mělo být při jednoporcentní státní půjčce zadrženo 50\% bývalého oběhu, povedlo se však jen 28,6\%. Tato půjčka byla kryta dávkou z majetku a progresivně stanovenou dávkou z přírůstku majetku za války. Díky těmto dávkám byla měna do roku 1927 podpořena 6,7 mld. Kč. Poměr československé ku rakousko-uherské koruně byl stanoven 1:1. Vůči sousedním zemím, které bojovaly s hyperinflací se Kč jevila jako velmi stabilní měna. Rašínem prosazovaná deflační politika, ale znamenala velkou ránu pro export, jenž v porovnání s předešlými roky klesl o více než 30\%. Celkově Rašínův příspěvek představoval velký úspěch pro československou hospodářskou politiku a kvalitu měny.
\\\indent Byl tu však problém s rozložením ekonomické, ale i společenské vyspělosti různých zemí Československa. Zatímco země Česká a Moravskoslezská byly centrem průmyslu a rozvoje tehdejšího Rakousko-Uherska, Slovensko společně s Podkarpatskou Rusí byly vůči těmto zemím značně zaostalé, jelikož se jednalo o země agrární. Vše výše uvedené velmi znesnadňovalo sjednocení Československa jako jednotného trhu. Dále malá kapacita vnitrostátního obchodu a velká kapacita průmyslu způsobila, že asi třetina veškeré produkce byla posílána na zahraniční trhy. Nově vzniklé země jako Českolslovenská republika a Rakousko zaujímaly důležité postavení tranzitních států ve střední Evropě. Spojeno s výše uvedenou potřebou vývozu ČSR potřebovala výkonnou dopravní infrastrukturu. ČSR však byla vnitrozemský stát bez přístupu k vlastnímu námořnímu přístavu, a tak se stala značně závislou na zahraničním obchodu s Německem nebo Itálií. Nejdůležitější roli v prvních letech sehráli železnice a říční plavba. Jenom samotné říční spojení s Hamburkem a Terstem pokrývaly 10 \% exportu a 12 \% importu československého zahraničního obchodu.
\\\indent Jak už bylo zmíněno ČSR byla geograficky uspůsobena pro fungování tranzitní země. Zvýšená potřeba samostatnosti nově vzniklých zemí, ovšem zapříčinila velice drahé celní polpatky. Dále kvůli ne úplně úspěšným konferencím byla tarifní situace též značně nepříznivá. Československý průmysl musel proto počítat s velkými náklady způsobenými průvozem zboží jinými státy. Pro ČSR bylo tedy důležité vyjednat výhodné tarifní smlouvy s Německem a ustanovení o tranzitu. Tento tranzit byl jedním z hlavních bodů programu ČSR na pařížské mírové konferenci v letech 1918-1919. Další nutný bod programu obsahoval získání přístupu k Dunaji, což se učinilo získáním hranice na Dunaji s Bratislavou. ČSR tím chtěla docílit odtržení hospodářských vztahů Německa s Maďarskem aby Maďarský obchod mohla ovlivňovat. Také pomocí Dunaje se ČSR propojila s Ruskem, Černým a Středozemním mořem. Dále pak následovala Labe jako hlavní spoj s Hamburským přístavem, který zprostředkovával export do západní Evropy a USA o objemu cca. 250 milionů korun ročně.
\\\indent Celkově měla ČSR na prvním místě internacionalizaci říčních toků (Dunaj, Labe, Visla) a internacionalizaci železničních sítí ve střední Evropě. }

\subsection{Politický kontext}

\paragraph{\quad ČSR po první světové válce 

Krizový stav mezinárodních dopravních styků po první světové válce si žádal rychlé řešení, aby bylo zamezeno větším propadům stavů ekonomik mnoha zemí střední Evropy. Prvním krokem se stal provizorní společný dopravní výbor, jenž vznikl 7. listopadu 1918 složený z ČSR, Maďarska, Rakouska, Polska a Slovinska. Tento zásah měl alespoň trochu uklidnit situaci ve střední Evropě, ale k požadovanéu uklidnění došlo až po řadě dalších konferencí. Jedním z dalších důežitých bodů byl 25. ledna 1919 vznik Komise pro mezinárodní reži přístavů, vodních cest a železnic. Hlavními členy této komise byla velká pětka (tedy USA, Velká Británie, Francie, Itálie a Japonsko), Československo zde bylo též zastoupeno Edvardem Benešem a Karlem Kramářem. Dvě subkomise se zabývali otázkami internacionalizace splavných toků a svobodou tranzitu. Na pařížské mírové konferenci vznikl návrh na zavedení všeobecného mezinárodního železničního režimu. Podle tohoto režimu měly být hlavní tratě prohlášeny za mezinárodní. Řešení všech těchto problémů bylo však dovoleno pouze státům dohodovým (vítězným) a neutrálním, proto byla dopravní komise v říjnu roku 1919 rozšířena o zástupce členských států nezúčastněných na pařížské mírové konferenci díky působení Společnosti národů. To dovolilo řádným konferencím ohledně svobodného tranzitu, řešení problému repartice vozového parku Rakouska-Uherska nebo oběhu zboží v Evropě.    }

\subsection{Železnice}

\paragraph{\quad Po první světové válce byla v evropě železniční síť důležitým tématem. Nejen že byly válkou poničené tratě samotné, ale chyběly vozy, uhlí a fungující systém mezi nově vytvořenými státy. Situace se jevila jako nestabilní.
\textit{(Proto mělo na Pařížské mírové konferenci jednání o dopravě prioritu. Rozpad Rakousko-Uherské monarchie společně se spoustou změn ohledně hranic si žádalo komplexní projednání týkající se tarifů a vozových parků. Z tohoto důvodu 7. listopadu 1918 vznikl Společný dopravní výbor pro Československou republiku, Polsko, Rakousko, Slovinsko a Ukrajinu.) }
\\\indent  Mimo tyto mezinárodní problémy měla ČSR však i mnoho dalších, vnitrostátních. Rozpad Rakouska-Uherska dal za vznik velmi protáhlé zemi, která na toto geografické rozložení nebyla dopravně vůbec uzpůsobena. Též velké rozdíly ekonomických poměrů Čech, Moravskoslezka, Slovenska a Podkarpatské Rusi vůbec nepřispívaly. Zatímco země Česká byla centrem průmyslu a rozvoje tehdejšího Rakousko-Uherska, Moravskoslezko, Slovensko společně s Podkarpatskou Rusí byly vůči těmto zemím ekonomicky zaostalé a zěmědělsky orientované. Proto se jednou z prvních zásadních reforem stala industrializace Slovenska. To však způsobilo ještě větší potřebu výstavby nových železničních tratí, které by uspokojily tyto důležité tepny. Další přispívající faktor byl ten, že v rámci Rakousko-Uherské monarchie zátěžové tahy, jež procházely země České a Moravskoslezské směřovaly ze severu na jih. Toto vše tedy způsobilo velice ambiciózní plány ČSR na výstavbu až 500 nových tratí, z nichž nakonec byl postaven pouze zlomek.
\\\indent Přece se, ale rada Českolslovenských drah (dále už jen ČSD) velmi snažila o prosazení internacionalizace železničních tratí. Myšlenka této internacionalizace mířila na obejítí Německa jako tranzitní země. Československý návrh též počítal s internacionalizací tratí ve východní Evropě směrem na Rusko mezi něž patřily sítě:
\\\indent a) Lyon-Miláno-Terst-Záhřeb-Bělehrad-Bukurešť-Oděssa
\\\indent b) Paříž-Štrasburk-Norimberk-Praha-Krakov-Varšava
\\\indent c) Terst-Bratislava-Praha
\\\indent d) Praha-Brno-Košice-Jassy-Oděssa-Bukurešť
\\\indent e) Praha-Bratislava-Budapešť-Bělehrad-Sofie
\\\indent f) Praha-Bratislava-Lvov-Kyjev-Moskva
\\\indent Z nichž bylo nejdůležitější prosadit tratě Bratislava-Terst a Bratislava-Záhřeb-Fiume, jelikož tyto tratě zajištovaly nejbližší cestu k námořním přístavům Terst a Fiume. Tímto by též mohlo být spojeno Českoslovesnko se Severní Itálií bez interference s Německým územím.



}










\end{document}
