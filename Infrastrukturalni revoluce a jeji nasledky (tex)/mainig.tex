\documentclass{article}
\usepackage{fullpage}
\usepackage[czech]{babel}
\usepackage{amsfonts}
\renewcommand{\baselinestretch}{1.5}
\title{\vspace{-2cm}Proč je důležitá dopravní infrastruktura pro růst ekonomiky a co znamenaly změny zátěžových směrů v Československu\vspace{-1.7cm}}
\date{}
\author{}

\begin{document}
\maketitle
\tableofcontents
\newpage
\section{Úvod}

\paragraph{\quad Doprava je každodenní součástí našich životů již nelze přehlédnout. Co se, ale skrývá za našimi zraky a jak je možné, že všechny produkty či lidé jsou tam kde chtějí být ve správný čas? Mohl by svět, jak ho známe fungovat bez tak rozvinuté infrastruktury? Odpověď je jednoduchá, nemohl. Infrastruktura se vyvíjí prakticky od počátku lidstva. Každá vyspělá společnost si zakládá na efektivní, vyspělé ekonomice. Za efektivní ekonomiku považujeme ekonomiku, která je schopna využít všech svých statků jež má k dispozici bez toho, aniž by obohacovala jednoho jedince na úkor někoho jiného. Jejíž základem jsou tři otázky co, jak a pro koho. Avšak bylo by záhodno zde přidat i otázky kam nebo čím. Upozaděn je totiž fakt, že do továren a obchodů musí být suroviny, produkty nebo pracovní síla nějak doručena. Mezi nabídkou a poptávkou nacházíme tedy třetí neutrální stranu: logistiku. To už však věděli i staří Římané. Římské impérium by se svým územím přes celou Evropu bez, na jeho dobu velice vyspělé, sítě silnic nemohlo fungovat. Pokud nahlédneme dále do historie můžeme si všimnout jistého válečného trendu. Vyhrává ten, kdo má fungující logistiku nebo ochromí logistiku nepřítele. Toto jsme mohli pozorovat nejen u Římanů, ale i Francie, Pruska, Anglie potom za druhé světové války při velkých bitvách na moři o zásobování, jenž nakonec způsobily pád Německa. Zkrátka není vyspělé ekonomiky bez vyspělé infrastruktury. Bohužel to stále spoustě lidem nedochází.
 }
\paragraph{\quad Tato práce se }


\section{Historie}

\paragraph{\quad Rozpad Rakouska-Uherska znamenal pro střední Evropu spoustu nových problémů, ale i příležitostí. Nově vzniklé země jako Českolslovenská republika a Rakousko zaujímaly důležité postavení tranzitních států ve střední Evropě. ČSR však byla vnitrozemský stát bez přístupu k vlastnímu námořnímu přístavu, a tak se stala značně závislou na zahraničním obchodu s Německem nebo Itálií. Jenom samotné říční spojení s Hamburkem a Terstem pokrývaly 10 \% exportu a 12 \% importu československého zahraničního obchodu.
\\\indent Jak už bylo zmíněno ČSR byla geograficky uspůsobena pro  fungování tranzitní země. Nejen to, ČSR byla též silně exportní země kde výrobní kapacita silně překračovala možnosti obchodu na národním trhu, což vedlo k uplatňení na zahraničním trhu. Československý průmysl musel proto počítat s velkými náklady způsobenými průvozem zboží jinými státy. Pro ČSR bylo tedy důležité vyjednat výhodné tarifní smlouvy s Německem a ustanovení o tranzitu. Tento tranzit byl jedním z hlavních bodů programu ČSR na pařížské mírové konferenci v letech 1918-1919. Další nutný bod programu obsahoval získání přístupu k Dunaji, což se učinilo získáním hranice na Dunaji s Bratislavou. ČSR tím chtěla docílit odtržení hospodářských vztahů Německa s Maďarskem aby Maďarský obchod mohla ovivňovat. Také pomocí Dunaje se ČSR propojila s Ruskem, Černým a Středozemním mořem. Dále pak následovala Labe jako hlavní spoj s Hamburským přístavem, který zprostředkovával export do západní Evropy a USA o objemu cca. 250 milionů korun ročně.
\\\indent Celkově měla ČSR na prvním místě internacionalizaci říčních toků (Dunaj, Labe, Visla) a internacionalizaci železničních sítí ve střední Evropě.  }

\subsection{Železnice}

\subsubsection{Roky 1918 - 1938}


\paragraph{\quad Po první světové válce byla v evropě železniční síť důležitým tématem. Nejen že byly válkou poničené tratě samotné, ale chyběly vozy, uhlí a fungující systém mezi nově vytvořenými státy. Situace se jevila jako nestabilní.
\textit{(Proto mělo na Pařížské mírové konferenci jednání o dopravě prioritu. Rozpad Rakousko-Uherské monarchie společně se spoustou změn ohledně hranic si žádalo komplexní projednání týkající se tarifů a vozových parků. Z tohoto důvodu 7. listopadu 1918 vznikl Společný dopravní výbor pro Československou republiku, Polsko, Rakousko, Slovinsko a Ukrajinu.) }
\\\indent  Mimo tyto mezinárodní problémy měla ČSR však i mnoho dalších, vnitrostátních. Rozpad Rakouska-Uherska dal za vznik velmi protáhlé zemi, která na toto geografické rozložení nebyla dopravně vůbec uzpůsobena. Též velké rozdíly ekonomických poměrů Čech, Moravskoslezka, Slovenska a Podkarpatské Rusi vůbec nepřispívaly. Zatímco země Česká a Moravskoslezská byly centrem průmyslu tehdejšího Rakousko-Uherska, Slovensko společně s Podkarpatskou Rusí byly vůči těmto zemím ekonomicky zaostalé a zěmědělsky orientované. Proto se jednou z prvních zásadních reforem stala industrializace Slovenska. To však způsobilo ještě větší potřebu výstavby nových železničních tratí, které by uspokojily tyto důležité tepny. Další přispívající faktor byl ten, že v rámci Rakousko-Uherské monarchie zátěžové tahy, jež procházely země České a Moravskoslezské směřovaly ze severu na jih. Toto vše tedy způsobilo velice ambiciózní plány ČSR na výstavbu až 500 nových tratí, z nichž nakonec byl postaven pouze zlomek.
\\\indent Přece se, ale rada Českolslovenských drah (dále už jen ČSD) velmi snažila o prosazení internacionalizace železničních tratí. Myšlenka této internacionalizace mířila na obejítí Německa jako tranzitní země. Československý návrh též počítal s internacionalizací tratí ve východní Evropě směrem na Rusko mezi něž patřily sítě:
\\\indent a) Lyon-Miláno-Terst-Záhřeb-Bělehrad-Bukurešť-Oděssa
\\\indent b) Paříž-Štrasburk-Norimberk-Praha-Krakov-Varšava
\\\indent c) Terst-Bratislava-Praha
\\\indent d) Praha-Brno-Košice-Jassy-Oděssa-Bukurešť
\\\indent e) Praha-Bratislava-Budapešť-Bělehrad-Sofie
\\\indent f) Praha-Bratislava-Lvov-Kyjev-Moskva
\\\indent Z nichž bylo nejdůležitější prosadit tratě Bratislava-Terst a Bratislava-Záhřeb-Fiume, jelikož tyto tratě zajištovaly nejbližší cestu k námořním přístavům Terst a Fiume. Tímto by též mohlo být spojeno Českoslovesnko s Severní Itálií bez interference s Německým územím. 
 }










\end{document}
